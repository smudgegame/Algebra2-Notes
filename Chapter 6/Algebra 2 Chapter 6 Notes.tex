\documentclass{report}
\usepackage{amsmath}
\usepackage[margin=0.75in]{geometry}
\usepackage{fancyhdr}
\usepackage{etoolbox}
\usepackage{amsthm}
\usepackage{calc}
\usepackage{tikz,pgfplots}
\usepackage{graphicx}
\usepackage{rotating}
\usetikzlibrary{arrows}
\usepackage{array,booktabs,arydshln,xcolor}
\usepgfplotslibrary{fillbetween}
\pgfplotsset{compat=1.16,width=10cm}


%% New colors

\definecolor{darkgreen}{rgb}{0.0, 0.4, 0.2}

%%Section and chapter references macros
%%%%%%%%%%%%%%%%%%%

\makeatletter
\newcommand*{\currentname}{\@currentlabelname}
\makeatother

\newcommand\getcurrentref[1]{%
 \ifnumequal{\value{#1}}{0}
  {??}
  {\the\value{#1}}%
}    

\newcommand{\longdivision}[2]{
    \settowidth{\dividendlength}{#1}
    \settowidth{\divisorlength}{#2}
    \settoheight{\dividendheight}{#1}
    \settoheight{\maxheight}{#1#2}
    \settoheight{\divisorheight}{#2}

    \begin{tikzpicture} [baseline=.5pt]
        \node at (-.5*\divisorlength-1pt,.5*\divisorheight) {#2};
        \node at (.5*\dividendlength+5pt,.5*\dividendheight) {#1};
        \draw [thick]  (0pt,-.22*\dividendheight) arc (-70:60:\maxheight*.41 and \maxheight*.82) -- ++(\dividendlength+7pt,0pt);
    \end{tikzpicture}
}

\newlength{\dividendlength}
\newlength{\divisorlength}
\newlength{\dividendheight}
\newlength{\divisorheight}
\newlength{\maxheight}

%%%%%%%Commands%%%%%%%%%%%%%%%%%%%%%
\theoremstyle{definition}
\newtheorem{example}{\bf Example}
\newtheorem{youtry}{\bf You Try It!}
\newtheorem{definition}{\bf Definition}[section]




%%%%%%%%%%%%%Header%%%%%%%%%%%%%%%%%%%
\pagestyle{fancy}
\fancyhf{}
\renewcommand{\headrulewidth}{0pt}
\lhead{}
\rhead{Algebra 2: Section  \getcurrentref{chapter}.\getcurrentref{section}}
\cfoot{ }
\rfoot{Page \thepage}
%%%%%%%%%%%%%%%%%%%%%%%%%%%%%%%%%%%%%%

%% Set start page here 
\setcounter{page}{1}



%%% Set Chapter counter here
\setcounter{chapter}{6}
\setcounter{section}{0}


%%%%%%%%%%%%%%% New Section Template %%%%%%%%%%%%%%%%
%%%%%%%%%%%%%%%%%%%%%%%%%%%%%%%%
%%%%%%%%%%%%%%%%%%%%%%%%%%%%%%%%
%%%%%%%   Section _._   %%%%%%%%
%%%%%%%%%%%%%%%%%%%%%%%%%%%%%%%%
%%%%%%%%%%%%%%%%%%%%%%%%%%%%%%%%
% \section{    NAME HERE   }
% \setcounter{example}{0}
% \setcounter{definition}{0}
% %%%%%%%%%%%%%%%%%%%%%%%%%%%%%%%%
% %%%%%%%%%%%%%%%%%%%%%%%%%%%%%%%%
% \begin{definition}
%     Definition
% \end{definition}
% \begin{example}
%     Example
% \end{example}
%
%\vspace{-0.5cm}
%\hspace{-0.5cm}
%%
%% LeftOfPage%%%%%
%\begin{minipage}{0.45\linewidth}
% \begin{itemize}
%     \item[(a)]
% \end{itemize}
%%
%\vspace{2.75cm}
%%
% \begin{itemize}
%     \item[(c)]
% \end{itemize}
%%
%\vspace{2.75cm}
%%
%\end{minipage}
%\hspace{1.5cm}
%%RightOfPage%%%%%
%\begin{minipage}{0.45\linewidth}
% \begin{itemize}
%     \item[(b)]
% \end{itemize}
%%
%\vspace{2.75cm}
%%
% \begin{itemize}
%     \item[(d)]
% \end{itemize}
%%
%\vspace{2.75cm}
%%
%\end{minipage}
%%%%%%%%%%%%%%%%%%%%%%
%
%%%%%%%%%%%%%%%%% Example 2
%\begin{example}
%     Use \textbf{elimination} to solve each system of equations.
%\end{example}
%%
%\vspace{-0.5cm}
%\hspace{-0.5cm}
%%
%% LeftOfPage%%%%%
%\begin{minipage}{0.45\linewidth}
% \begin{itemize}
%     \item[(a)]
% \end{itemize}
%%
%\vspace{2.75cm}
%%
% \begin{itemize}
%     \item[(c)]
% \end{itemize}
%%
%\vspace{2.75cm}
%%
%\end{minipage}
%\hspace{1.5cm}
%%RightOfPage%%%%%
%\begin{minipage}{0.45\linewidth}
% \begin{itemize}
%     \item[(b)]
% \end{itemize}
%%
%\vspace{2.75cm}
%%
% \begin{itemize}
%     \item[(d)]
% \end{itemize}
%%
%\vspace{2.75cm}
%%
%\end{minipage}
%%%%%%%%%%%%%%%%%%%%%
%\vfill
% \noindent\fbox{\large\textbf{_._ Homework}: page \small }
%%%%%%%%%%%%%%%%%%%%%%%%%%%%%%%%%%%%%%
%%%%%%%%%%%%%%%%%%%%%%%%%%%%%%%%%%%%%%
% \newpage
%%%%%%%%%%%%%%%%%%%%%%%%%%%%%%%%%%%%%%
%%%%%%%%%%%%%%%%%%%%%%%%%%%%%%%%%%%%%%



% number line example%%%%
%\begin{tikzpicture}
        %\draw[latex-latex] (-3.5,0) -- (3.5,0) ; %edit here for the axis
        %\foreach \x in  {-3,-2,-1,0,1,2,3} % edit here for the vertical lines
        %\draw[shift={(\x,0)},color=black] (0pt,3pt) -- (0pt,-3pt);
        %\foreach \x in {-3,-2,-1,0,1,2,3} % edit here for the numbers
        %\draw[shift={(\x,0)},color=black] (0pt,0pt) -- (0pt,-3pt) node[below] 
        %{$\x$};
        %\draw[*-o] (0.92,0) -- (2.08,0);
        %\draw[very thick] (0.92,0) -- (1.92,0);
%\end{tikzpicture}



\setcounter{section}{-1}
\begin{document}

\noindent\LARGE\textbf{Chapter 6 Polynomial Functions}\normalsize

%%%%%%%%%%%%%%%%%%%%%%%%%%%%%%%
%%%%%%%%%%%%%%%%%%%%%%%%%%%%%%%
%%%%%%   Section 6.0   %%%%%%%%
%%%%%%%%%%%%%%%%%%%%%%%%%%%%%%%
%%%%%%%%%%%%%%%%%%%%%%%%%%%%%%%
\rhead{Algebra 2: Section  \getcurrentref{chapter}.0}
 \section{Pre-assessment }
 \setcounter{section}{0}
 \setcounter{example}{0}
 \setcounter{definition}{0}
 %%%%%%%%%%%%%%%%%%%%%%%%%%%%%%%%
 %%%%%%%%%%%%%%%%%%%%%%%%%%%%%%%%
 \vspace{0.5cm}
 
\textbf{Match each of the vocabulary terms on the left with the appropriate letter and definition on the right.}\\

\begin{enumerate}
	\begin{minipage}[t]{0.45\linewidth}
	
			\item coefficent
				\vspace{0.25cm}
		\item like terms
				\vspace{0.25cm}
			\item root of an equation
				\vspace{0.25cm}
			\item $x$-intercept
				\vspace{0.25cm}
			\item maximum of a function
				\vspace{0.25cm}
	
	\end{minipage}
	\begin{minipage}[t]{0.45\linewidth}
		\begin{itemize}
			\item[A.] the $y$-value of the highest point of the graph of the function
				\vspace{0.25cm}
			\item[B.] the horizontal number line that divides the coordinate plane
				\vspace{0.25cm}
			\item[C.] the numerical factor in a term
				\vspace{0.25cm}
			\item[D.] a value of the variable that makes the equation true
				\vspace{0.25cm}
			\item[E.] terms that contain the same variables raised to the same powers
				\vspace{0.25cm}
			\item[F.] the $x$-coordinate of a point where the graph intersects the $x$-axis.
				\vspace{0.25cm}
		\end{itemize}
	\end{minipage}
\end{enumerate}
\noindent \textbf{Evaluate each expression.}\\

\begin{enumerate}
\setcounter{enumi}{5}
	\begin{minipage}[t]{0.225\linewidth}
	\item $6^4$
	\end{minipage}
	\begin{minipage}[t]{0.225\linewidth}
	\item $-5^4$
	\end{minipage}
	\begin{minipage}[t]{0.225\linewidth}
	\item $(-1)^5$
	\end{minipage}
	\begin{minipage}[t]{0.225\linewidth}
	\item $\displaystyle\Bigg{(}-\frac{2}{3}\Bigg{)}^2$
	\end{minipage}
\end{enumerate}

\noindent \textbf{Evaluate each expression for the given value of the variable.}\\

\begin{enumerate}
\setcounter{enumi}{9}
	\begin{minipage}[t]{0.45\linewidth}
		\item $x^4-5x^2-6x-8$ for $x=3$
			\vspace{1cm}
		\item $2x^3-3x^2-29x-30$ for $x=-2$
			\vspace{1cm}
	\end{minipage}
	\begin{minipage}[t]{0.45\linewidth}
		\item $2x^3-x^2-8x+4$ for $x=\displaystyle\frac{1}{2}$
			\vspace{1cm}
		\item $3x^4+5x^3+6x^2+4x-1$ for $x=-1$
			\vspace{1cm}
	\end{minipage}
\end{enumerate}

\noindent \textbf{Multiply or divide.}\\

\begin{enumerate}
\setcounter{enumi}{13}
	\begin{minipage}[t]{0.225\linewidth}
	\item $2x^3y\cdot 4x^2$
	\end{minipage}
	\begin{minipage}[t]{0.225\linewidth}
	\item $-a^2b\cdot ab^4$
	\end{minipage}
	\begin{minipage}[t]{0.225\linewidth}
	\item $\displaystyle \frac{-7t^4}{3t^2} $
	\end{minipage}
	\begin{minipage}[t]{0.225\linewidth}
	\item $\displaystyle\frac{3p^3q^2r}{12pt^4}$
	\end{minipage}
\end{enumerate}
\vfill
\begin{flushright}
\rotatebox{180}{\color{red}10. 10 11. 0 12. 0 13. $-1$ 14. $8x^5y$ 15. $-a^3b^5$ 16. $-\frac{7}{3}t^2$ 17. $\frac{p^2q^2r}{4t^4}$   }\\
\vspace{0.5cm}
\rotatebox{180}{\color{red}1. C 2. E 3. D  4. F 5. A 6. 1296 7. -625  8. -1  9. 4/9 \color{black}}
\end{flushright}
	
%%%%%%%%%%%%%%%%%%%%%%%%%%%%%%%%%%%%%
%%%%%%%%%%%%%%%%%%%%%%%%%%%%%%%%%%%%%
 \newpage
%%%%%%%%%%%%%%%%%%%%%%%%%%%%%%%%%%%%%
%%%%%%%%%%%%%%%%%%%%%%%%%%%%%%%%%%%%%

%%%%%%%%%%%%%%%%%%%%%%%%%%%%%%%
%%%%%%%%%%%%%%%%%%%%%%%%%%%%%%%
%%%%%%   Section 6.1   %%%%%%%%
%%%%%%%%%%%%%%%%%%%%%%%%%%%%%%%
%%%%%%%%%%%%%%%%%%%%%%%%%%%%%%%
\rhead{Algebra 2: Section  \getcurrentref{chapter}.\getcurrentref{section}}
\setcounter{section}{0}
 \section{  Polynomials }
 \indent\hfill\small\noindent \textbf{Objective}: Identify and classify polynomials \normalsize\\
 \vspace{-0.5cm}
 \setcounter{example}{0}
 \setcounter{definition}{0}
 %%%%%%%%%%%%%%%%%%%%%%%%%%%%%%%%
 %%%%%%%%%%%%%%%%%%%%%%%%%%%%%%%%
\begin{definition}
 A \textbf{monomial} is a number or a product of numbers and variables with whole number exponents. A \textbf{polynomial} is a monomial or a sum or difference of monomials. The \textbf{degree of a monomial} is the sum of the exponents of the variables.
\end{definition}

\begin{tabular}{p{4cm}p{1.75cm}p{1.75cm}p{1.75cm}p{1.75cm}p{1.75cm}}
\noindent\textbf{Polynomials:}& $3x^4$ & $2z^{12}+9z^3$ & $\displaystyle \frac{1}{2} a^7$ & $0.15x^{101}$ & $3t^2-t^3$ \\

\vspace{0.25cm}\noindent\textbf{Not Polynominals:} & \vspace{0.25cm}$3^x$ & \vspace{0.25cm}$|2b^3-6b|$ & \vspace{0.25cm}$\displaystyle \frac{8}{5y^2}$ & \vspace{0.25cm}$\displaystyle\frac{1}{2}\sqrt{x}$ & \vspace{0.25cm}$m^{0.75}-m$
\end{tabular}
\begin{example}
Identify the degree of each monomial.
\end{example}

\begin{itemize}
	\begin{minipage}[t]{0.45\linewidth}
		\item[(a)] $x^4$\\
		
		\item[(b)] $12$		
	\end{minipage}
	\begin{minipage}[t]{0.45\linewidth}
		\item[(c)] $4a^2b$\\
		
		\item[(d)] $x^3y^4z$
	\end{minipage}
\end{itemize}


\begin{definition}
The \textbf{degree of a polynomial} is given by the term with the greatest degree. A polynomial is in standard when its terms are written in decending order of degree. The \textbf{leading coefficent} the coefficent of the first term in standard form.
\end{definition}

\large
\begin{center}
$\color{red}\mathbf{5}\color{black}x^{\color{blue}\mathbf{3}\color{black}} +8x^2+3x-17$
\end{center}
\normalsize


\begin{definition}
A polynomial with two terms is called a \textbf{binomial}, and a polynomial with three terms is called a \textbf{trinomial}.
\end{definition}

\begin{center}	
	\begin{tabular}{|l|c|c|}
		\hline
		\multicolumn{3}{|c|}{\textbf{Classifying Polynomials by Degree}}\\
		\hline
		\multicolumn{1}{|c|}{\textbf{Name}} & \textbf{Degree} & \textbf{Example} \\
		\hline
		Constant&0& $-9$\\
		\hline
		Linear&1& $x-4$\\
		\hline
		Quadratic&2& $x^2+3x-1$\\
		\hline
		Cubic&3& $x^3+2x^2+x+1$\\
		\hline
		Quartic&4& $2x^4+x^3+3x^2+4x-1$\\
		\hline
		Quintic&5& $7x^5+x^4-x^3+3x^2+2x-1$\\
		\hline
	\end{tabular}
\end{center}

\begin{example}
Rewrite each polynomial in standard form. The identify the leading coefficent, degree, and number of terms. Name the polynomial.
\end{example}

\begin{minipage}[t]{0.45\linewidth}
\begin{itemize}
\item [(a.)] $2x+4x^3-1$ \vspace{0.25cm}\\
\vspace{0.25cm}
Standard Form:\\
\vspace{0.25cm}
Leading Coefficent:\\
\vspace{0.25cm}
Degree:\\
\vspace{0.25cm}
Terms:\\
\vspace{0.25cm}
Name:\\
\vspace{0.25cm}
\end{itemize}
\end{minipage}
\begin{minipage}[t]{0.45\linewidth}
\begin{itemize}
\item [(b.)] $7x^3-11x+x^5-2$ \vspace{0.25cm}\\
\vspace{0.25cm}
Standard Form:\\
\vspace{0.25cm}
Leading Coefficent:\\
\vspace{0.25cm}
Degree:\\
\vspace{0.25cm}
Terms:\\
\vspace{0.25cm}
Name:\\
\vspace{0.25cm}
\end{itemize}
\end{minipage}


\vspace{-0.75cm}
\begin{example}
Add or subtract. Write you answer in standard form.
\end{example}

\begin{minipage}[t]{0.45\linewidth}
	\begin{itemize}
		\item[(a.)] $(3x^2+7+x)+(14x^3+2+x^2-x)$
	\end{itemize}
\end{minipage}\
\begin{minipage}[t]{0.45\linewidth}
	\begin{itemize}
		\item[(b.)] $(1-x^2)-(3x^2+2x-5)$
	\end{itemize}
\end{minipage}

\vfill

 \noindent\fbox{\large\textbf{6.1 (day 1) Homework}: page 410 1-13 all Adv. 47-49 \small } \hfill 
%%%%%%%%%%%%%%%%%%%%%%%%%%%%%%%%%%%%%
%%%%%%%%%%%%%%%%%%%%%%%%%%%%%%%%%%%%%
 \newpage
%%%%%%%%%%%%%%%%%%%%%%%%%%%%%%%%%%%%%
%%%%%%%%%%%%%%%%%%%%%%%%%%%%%%%%%%%%%

\noindent\Large{\textbf{Polynomials (day 2)}}  \indent\hfill\small\noindent \textbf{Objective}: Evaluate and Graph Polynomials \normalsize\\

\begin{youtry}
Add or subtract. Write your answer in standard form.
\end{youtry}

\begin{minipage}[t]{0.45\linewidth}
(a) $(-36x^2+6x-11)+(6x^2+16x^3-5)$
\end{minipage}
\hfill
\begin{minipage}[t]{0.45\linewidth}
(b) $(5x^3+12+6x^2)+(15x^2+3x-2)$
\end{minipage}
\vfill
\begin{example}
Cardiac output is the amount of blood pumped through the heard. The output is measured by a technique called dye dilution. A doctor injects dye into a vein near the heart and measured the amount of dye in the arteries over time.\\

\noindent The cardiac output of a particular patient can be approximated by the function
\[f(t) = 0.0056t^3-0.22t^2+2.33t,\]
where $f(t)$ represents the concentration  of dye (in milligrams per liter).
\end{example}
\begin{itemize}
	\item[(a)] Evaluate $f(t)$ for $t=0$ and $t=3$.
	\vfill
	\item[(b)] Describe what the values of the function in part (a) represent.
	\vfill
\end{itemize}

\begin{example}
Graph each polynomial on a graphing calculator. Describe the graph, and identify the number of real zeros.
\end{example}


\begin{minipage}[t]{0.45\linewidth}
	\begin{itemize}
		\item[(a)] $f(x)=x^3-x$
		\vspace{1cm}
		\item[(c)] $h(x)=x^4-8x^2+1$
	\end{itemize}
\end{minipage}
\begin{minipage}[t]{0.45\linewidth}
	\begin{itemize}
		\item[(b)] $f(x)=-3x^3+2x+1$
		\vspace{1cm}
		\item[(d)] $k(x)=x^4+x^3-x^2+2x-3$
	\end{itemize}
\end{minipage}
\vfill


 \noindent\fbox{\large\textbf{6.1 (day 2) Homework}: page 410 31-39 all Adv. 41, 42 \small } \hfill 

%%%%%%%%%%%%%%%%%%%%%%%%%%%%%%%%%%%%%
%%%%%%%%%%%%%%%%%%%%%%%%%%%%%%%%%%%%%
 \newpage
%%%%%%%%%%%%%%%%%%%%%%%%%%%%%%%%%%%%%
%%%%%%%%%%%%%%%%%%%%%%%%%%%%%%%%%%%%%

%%%%%%%%%%%%%%%%%%%%%%%%%%%%%%%
%%%%%%%%%%%%%%%%%%%%%%%%%%%%%%%
%%%%%%   Section 6.2   %%%%%%%%
%%%%%%%%%%%%%%%%%%%%%%%%%%%%%%%
%%%%%%%%%%%%%%%%%%%%%%%%%%%%%%%
 \section{ Multiplying Polynomials }
 \indent\hfill\small\noindent \textbf{Objective}: To Multiply Polynomials and Binomial Expansion\normalsize\\
 \setcounter{example}{0}
 \setcounter{definition}{0}
 %%%%%%%%%%%%%%%%%%%%%%%%%%%%%%%%
 %%%%%%%%%%%%%%%%%%%%%%%%%%%%%%%%

\vspace{-0.5cm}

 \begin{example}
  Find each product.
 \end{example}
\begin{minipage}[t]{0.45\linewidth}
	\begin{itemize}
		\item[(a)] $3x^2(x^3+4)$
	\end{itemize}
\end{minipage}
\begin{minipage}[t]{0.45\linewidth}
	\begin{itemize}
		\item[(b)] $ab(a^3+3ab^2-b^3)$
	\end{itemize}
\end{minipage}

\vfill

\begin{example}
Find each product.
\end{example}
\begin{minipage}[t]{0.45\linewidth}
	\begin{itemize}
		\item[(a)] $(x-2)(1+3x-x^2)$
	\end{itemize}
\end{minipage}
\begin{minipage}[t]{0.45\linewidth}
	\begin{itemize}
		\item[(b)] $(x^2+3x-5)(x^2-x+1)$
	\end{itemize}
\end{minipage}
\vfill

\noindent\large \textbf{Binomial Expansion}\normalsize
\begin{example}
Find the product.
\end{example}
$(x+y)^3$
\vfill

%%%%%%%%%%%%%%%%%%%%%%%%%%%%%%%%%%%%%
%%%%%%%%%%%%%%%%%%%%%%%%%%%%%%%%%%%%%
 \newpage
%%%%%%%%%%%%%%%%%%%%%%%%%%%%%%%%%%%%%
%%%%%%%%%%%%%%%%%%%%%%%%%%%%%%%%%%%%%

\begin{center}
	\begin{tabular}{|lc|c|}
	\hline
	&&\textbf{Pascal's Triangle}\\
	\multicolumn{2}{|c|}{\textbf{Binomial Expansion}}& \textbf{(Coefficients)}\\
	\hline
	&&\\
	$(a+b)^0=$&\color{red}1&$\color{red}1\color{black}$\\
	\hline
	&&\\
	$(a+b)^1=$&$\color{red}1\color{black}a+\color{red}1\color{black}b$&$\color{red}1\quad1\color{black}$\\
	\hline
	&&\\
	$(a+b)^2=$&$\color{red}1\color{black}a^2+\color{blue}2\color{black}ab+\color{red}1\color{black}b^2$&$\color{red}1\quad \color{blue}2 \quad \color{red}1\color{black} $\\
	\hline
	&&\\
	$(a+b)^3=$&$\color{red}1\color{black}a^3+\color{blue}3\color{black}a^2b+\color{blue}3\color{black}ab^2+\color{red}1\color{black}b^3$&$\color{red}1\quad\color{blue} 3\quad 3\quad\color{red}1\color{black}$\\
	\hline
	&&\\
	$(a+b)^4=$&$\color{red}1\color{black}a^4+\color{blue}4\color{black}a^3b+\color{blue}6\color{black}a^2b^2+\color{blue}4\color{black}ab^3+\color{red}1\color{black}b^4$&$\color{red}1\quad\color{blue}4\quad6\quad4\quad\color{red}1\color{black}$\\
	\hline
	&&\\
	$(a+b)^5=$&$\color{red}1\color{black}a^5+\color{blue}5\color{black}a^4b+\color{blue}10\color{black}a^3b^2+\color{blue}10\color{black}a^2b^3+\color{blue}5\color{black}ab^4+\color{red}1\color{black}b^5$&$\color{red}1\quad\color{blue}5\quad10\quad10\quad5\quad\color{red}1\color{black}$\\
	\hline
	\end{tabular}
\end{center}

\begin{example}
Expand each expression using Pascal's triangle.
\end{example}

\begin{minipage}[t]{0.5\linewidth}
	\begin{itemize}
		\item[(a)] $(y-3)^4$
	\end{itemize}
\end{minipage}
\vfill
\begin{minipage}[t]{0.5\linewidth}
	\begin{itemize}
		\item[(b)] $(4z+5)^3$
	\end{itemize}
\end{minipage}

\vfill


 \noindent\fbox{\large\textbf{6.2 Homework}: page 418 1-13 odd, Adv. 52, 56  \small }
%%%%%%%%%%%%%%%%%%%%%%%%%%%%%%%%%%%%%
%%%%%%%%%%%%%%%%%%%%%%%%%%%%%%%%%%%%%
 \newpage
%%%%%%%%%%%%%%%%%%%%%%%%%%%%%%%%%%%%%
%%%%%%%%%%%%%%%%%%%%%%%%%%%%%%%%%%%%%

%%%%%%%%%%%%%%%%%%%%%%%%%%%%%%%
%%%%%%%%%%%%%%%%%%%%%%%%%%%%%%%
%%%%%%   Section 6.3   %%%%%%%%
%%%%%%%%%%%%%%%%%%%%%%%%%%%%%%%
%%%%%%%%%%%%%%%%%%%%%%%%%%%%%%%
 \section{ Dividing Polynomials  }
  \indent\hfill\small\noindent \textbf{Objective}: Use long and synthetic division to divide polynomials.  \normalsize\\
 \setcounter{example}{0}
 \setcounter{definition}{0}
 %%%%%%%%%%%%%%%%%%%%%%%%%%%%%%%%
 %%%%%%%%%%%%%%%%%%%%%%%%%%%%%%%%
 
\hspace{-0.5cm}\begin{minipage}[t]{0.45\linewidth}
	 \begin{example}
	 Divide using arithmetic long division.
	 \end{example}
	 \begin{itemize}
	 	\item[(a)] \longdivision{$277$}{$12$}
	 \end{itemize}
\end{minipage}
\hspace{1.25cm}
 \begin{minipage}[t]{0.45\linewidth}
 	 \begin{youtry}
 	 Divide.
 	 \end{youtry}
	 \begin{itemize}
	 	\item[(b)] \longdivision{$347$}{$8$}
	 \end{itemize}
\end{minipage}
 \vfill
 \begin{example}
 Divide using long division.
 \end{example}
 
\begin{minipage}[t]{0.45\linewidth}
(a) \, $(4x^2+3x^3+10)\div (x-2)$
\end{minipage}
 \hfill
 \begin{minipage}[t]{0.45\linewidth}
(b) \, $(15x^2+8x-12)\div (3x+1)$
\end{minipage}
\vfill
 \vfill
 
 \begin{example}
 Divide using synthetic division.
 \end{example}
\begin{minipage}[t]{0.45\linewidth}
(a) \,$(4x^2-12x+9)\div \bigg{(}x+\displaystyle\frac{1}{2}\bigg{)}$
\end{minipage}
 \hfill
\begin{minipage}[t]{0.45\linewidth}

(b) \,$(6x^2-5x-6)\div (x+3)$
\end{minipage}
\vfill
\begin{example}
Use synthetic substitution to evaluate the polynomial for the given value.
\end{example}

\begin{minipage}[t]{0.45\linewidth}
(a) \,$P(x)=x^3-4x^2+3x-5$ for $x=4$
\end{minipage}
 \hfill
\begin{minipage}[t]{0.45\linewidth}
(b) \,$P(x)=4x^4+2x^3+3x+5$ for $x=\displaystyle-\frac{1}{2}$
\end{minipage}
\vfill

 
 \noindent\fbox{\large\textbf{6.3 (day 1) Homework}: page 426 \, 1-12 all, Adv. 53, 57 \small }
 
 
%%%%%%%%%%%%%%%%%%%%%%%%%%%%%%%%%%%%%
%%%%%%%%%%%%%%%%%%%%%%%%%%%%%%%%%%%%%
 \newpage
%%%%%%%%%%%%%%%%%%%%%%%%%%%%%%%%%%%%%
%%%%%%%%%%%%%%%%%%%%%%%%%%%%%%%%%%%%%

\noindent\Large{\textbf{6.3 (day 2)}} 
  \indent\hfill\small\noindent \textbf{Objective}: Use long and synthetic division to divide polynomials.  \normalsize\\

 
 \begin{youtry}
 Divide using long division.
 \end{youtry}
 
\begin{minipage}[t]{0.45\linewidth}
(a) \, $(2x^2+7x+7)\div (x+2)$
\end{minipage}
 \hfill
 \begin{minipage}[t]{0.45\linewidth}
(b) \, $(x^2+5x-28)\div (x-3)$
\end{minipage}
\vfill
\vfill
 
 \begin{example}
 Divide using synthetic division.
 \end{example}
\begin{minipage}[t]{0.45\linewidth}
(a) \,$(x^2-3x-18)\div (x-6)$
\end{minipage}
 \hfill
\begin{minipage}[t]{0.45\linewidth}

(b) \,$(x^4-7x^3+9x^2-22x+25)\div (x+3)$
\end{minipage}
\vfill

\begin{center}
	\begin{tabular}{|l|l|}
		\hline
		\multicolumn{2}{|c|}{}\\
		\multicolumn{2}{|c|}{\large\textbf{Remainder Theorem}\normalsize}\\ 
		\hline
		&\\
		\multicolumn{1}{|c|}{\textbf{Theorem}} & \multicolumn{1}{c|}{\textbf{Example}}\\
		\hline
		&\\
		If the polynomial function $P(x)$ is divided by $x-\mathbf{a}$,& Divide $x^3-4x^2+5x+1$ by $x-3$\\
		 then the remainder $r$ is $P(\mathbf{a})$.  & 
		 \multicolumn{1}{c|}{$
			\renewcommand\arraystretch{1.5}
			\setlength\doublerulesep{0pt}
			\begin{array}{rrrrr}
				\multicolumn{1}{r|}{\color{red}\mathbf{3}\color{black}} & 1 & -4 & 5 & 1\\\cline{0-0}
			 	& \downarrow & 3& -3 & 6 \\\cline{2-5}
			 	& 1 & -1 & 2 & \multicolumn{1}{|r}{\color{green}\mathbf{7}\color{black}}\\\cline{5-5}
			\end{array}
		$}\\
				&\\
		&\multicolumn{1}{c|}{$P(\color{red}\mathbf{3}\color{black})=\color{green}\mathbf{7}\color{black}$}\\
		&\\
		\hline
	\end{tabular}
\end{center}

\begin{example}
Use synthetic substitution to evaluate the polynomial for the given value.
\end{example}

\begin{minipage}[t]{0.45\linewidth}
(a) \,$P(x)=x^3+3x^2+4$ for $x=-3$
\end{minipage}
 \hfill
\begin{minipage}[t]{0.45\linewidth}
(b) \,$P(x)=5x^2+9x+3$ for $x=\displaystyle\frac{1}{5}$
\end{minipage}
\vfill

 
 \noindent\fbox{\large\textbf{6.3 (day 2) Homework}: page 426 \, 13, 17, 19, 23, 25, 27, 43, 44 \small }
 
 
%%%%%%%%%%%%%%%%%%%%%%%%%%%%%%%%%%%%%
%%%%%%%%%%%%%%%%%%%%%%%%%%%%%%%%%%%%%
 \newpage
%%%%%%%%%%%%%%%%%%%%%%%%%%%%%%%%%%%%%
%%%%%%%%%%%%%%%%%%%%%%%%%%%%%%%%%%%%%


\rhead{Algebra 2: Section  \getcurrentref{chapter}.2  \&  \getcurrentref{chapter}.3  Review }
\noindent\Large{\textbf{6.2 \& 6.3 Review}}  \indent\hfill\small\noindent \textbf{Objective}: Multiply and Divide Polynomials \large\\

\noindent Find each product. \\

\begin{enumerate}
\begin{minipage}[t]{0.45\linewidth}
\item  $3x^2(2x^2+9x-6)$
\end{minipage}
\hfill
\begin{minipage}[t]{0.45\linewidth}
\item $(2x+5y)(3x^2-4xy+2y^2)$
\end{minipage}
\end{enumerate}
\vfill

\noindent Expand each expression. (Use Pascal's triangle)\\

\begin{enumerate}
\setcounter{enumi}{2}
\begin{minipage}[t]{0.45\linewidth}
\item  $(x-3y)^3$
\end{minipage}
\hfill
\begin{minipage}[t]{0.45\linewidth}
\item $(x-2)^5$
\end{minipage}
\end{enumerate}
\vfill

\noindent Divide.\\

\begin{enumerate}
\setcounter{enumi}{4}
\begin{minipage}[t]{0.45\linewidth}
\item  \longdivision{$647$}{$7$}
\end{minipage}
\hfill
\begin{minipage}[t]{0.45\linewidth}
\item \longdivision{$3452$}{$9$}
\end{minipage}
\end{enumerate}
\vfill
\vfill
\vfill

\noindent Use long division to divide the polynomials. Write as Quotient + Remainder/Divisor.  \\

\begin{enumerate}
\setcounter{enumi}{6}
\begin{minipage}[t]{0.45\linewidth}
\item  $(2x^2+3x-20)\div(x-2)$
\end{minipage}
\hfill
\begin{minipage}[t]{0.45\linewidth}
\item $(x^4+6x^3+6x^2)\div(x+5)$
\end{minipage}
\end{enumerate}
\vfill
\vfill
\vfill


%%%%%%%%%%%%%%%%%%%%%%%%%%%%%%%%%%%%%
%%%%%%%%%%%%%%%%%%%%%%%%%%%%%%%%%%%%%
 \newpage
%%%%%%%%%%%%%%%%%%%%%%%%%%%%%%%%%%%%%
%%%%%%%%%%%%%%%%%%%%%%%%%%%%%%%%%%%%%


\noindent Use synthetic division to divide the polynomials. Write as Quotient + Remainder/Divisor.  \\

\begin{enumerate}
\setcounter{enumi}{8}
	\begin{minipage}[t]{0.45\linewidth}
		\item  $x^4-3x^3-7x-14)\div(x-4)$
	\end{minipage}
	\hfill
	\begin{minipage}[t]{0.45\linewidth}
		\item $(x^2+9x+6)\div(x+8)$
	\end{minipage}
\end{enumerate}
\vfill



\noindent Use synthetic substitution (The Remainder Theorem) to evaluate the polynomial for the given value.\\

\begin{enumerate}
\setcounter{enumi}{10}
	\begin{minipage}[t]{0.45\linewidth}
		\item \,$P(x)=4x^3-5x^2-x+2$ for $x=-1$
	\end{minipage}
	 \hfill
	\begin{minipage}[t]{0.45\linewidth}
		\item \,$P(x)=25x^2-16$ for $x=\displaystyle\frac{4}{5}$
	\end{minipage}
\end{enumerate}
\vfill

\begin{enumerate}
\setcounter{enumi}{12}
	\begin{minipage}[t]{0.45\linewidth}
		\item \,$P(x)=4x^3-5x^2-x+2$ for $x=-1$
	\end{minipage}
	 \hfill
	\begin{minipage}[t]{0.45\linewidth}
		\item \,$P(x)=25x^2-16$ for $x=\displaystyle\frac{4}{5}$
	\end{minipage}
\end{enumerate}
\vfill


%%%%%%%%%%%%%%%%%%%%%%%%%%%%%%%%%%%%%
%%%%%%%%%%%%%%%%%%%%%%%%%%%%%%%%%%%%%
 \newpage
%%%%%%%%%%%%%%%%%%%%%%%%%%%%%%%%%%%%%
%%%%%%%%%%%%%%%%%%%%%%%%%%%%%%%%%%%%%
\rhead{Algebra 2: Section  \getcurrentref{chapter}.\getcurrentref{section}}
%%%%%%%%%%%%%%%%%%%%%%%%%%%%%%%
%%%%%%%%%%%%%%%%%%%%%%%%%%%%%%%
%%%%%%   Section 6.4   %%%%%%%%
%%%%%%%%%%%%%%%%%%%%%%%%%%%%%%%
%%%%%%%%%%%%%%%%%%%%%%%%%%%%%%%
 \section{Factoring Polynomials}
 \indent\hfill\small\noindent \textbf{Objective}:  Use the Factor Theorem to determine factors of a polynomial. \normalsize\\
 \setcounter{example}{0}
 \setcounter{definition}{0}
 %%%%%%%%%%%%%%%%%%%%%%%%%%%%%%%%
 %%%%%%%%%%%%%%%%%%%%%%%%%%%%%%%%
 
 \begin{center}
	\begin{tabular}{|l|l|}
		\hline
		\multicolumn{2}{|c|}{}\\
		\multicolumn{2}{|c|}{\large\textbf{Factor Theorem}\normalsize}\\ 
		\hline
		&\\
		\multicolumn{1}{|c|}{\textbf{Theorem}} & \multicolumn{1}{c|}{\textbf{Example}}\\
		\hline
		&\\
		For any polynomial $P(x)$, $(x-a)$ is a factor & Because $P(1)=1^2-1=0$, $(x-1)$\\
		of $P(x)$ if and only if $P(a)=0$. & is a factor of $P(x)=x^2-1$.\\
		\hline
	\end{tabular}
\end{center}
 
\begin{example}
 Determine whether the given binomial is a factor of the polynomial $P(x)$.
\end{example}
\begin{minipage}[t]{0.45\linewidth}
 (a) $(x-3)$; $P(x)=x^2+2x-3$
\end{minipage}
\begin{minipage}[t]{0.45\linewidth}
 (b) $(x+4)$; $P(x)=2x^4+8x^3+2x+8$
\end{minipage}
\vfill

\begin{example}
Factor by grouping.
\end{example}
\begin{minipage}[t]{0.45\linewidth}
 (a) $x^3+3x^2-4x-12$
\end{minipage}
\begin{minipage}[t]{0.45\linewidth}
 (b) $x^3-2x^2-9x+18$
\end{minipage}
\vfill
\begin{youtry}
Factor by grouping
\end{youtry}
\begin{minipage}[t]{0.45\linewidth}
 (a) $2x^3+x^2+8x+4$
\end{minipage}
\begin{minipage}[t]{0.45\linewidth}
 (b) $8y^3-4y^2-50y+25$
\end{minipage}
\vfill

\vfill
 \noindent\fbox{\large\textbf{6.4 (day 1) Homework}: page 433 \, 1-9 all, 17-19, Adv. 41, 43  \small }
%%%%%%%%%%%%%%%%%%%%%%%%%%%%%%%%%%%%%
%%%%%%%%%%%%%%%%%%%%%%%%%%%%%%%%%%%%%
 \newpage
%%%%%%%%%%%%%%%%%%%%%%%%%%%%%%%%%%%%%
%%%%%%%%%%%%%%%%%%%%%%%%%%%%%%%%%%%%%

\noindent\Large{\textbf{6.4 (day 2) Factoring}} 
\indent\hfill\small\noindent \textbf{Objective}: Factor the sum and difference of two cubes.  \normalsize\\


 \begin{center}
	\begin{tabular}{|l|c|}
		\hline
		\multicolumn{2}{|c|}{}\\
		\multicolumn{2}{|c|}{\large\textbf{Factoring The Sum and Difference of Two Cubes}\normalsize}\\ 
		\hline
		&\\
		\multicolumn{1}{|c|}{\textbf{Method}} & \multicolumn{1}{c|}{\textbf{Algebra}}\\
		\hline
		&\\
		Sum of two cubes& $a^3\color{blue} \mathbf{+} \color{black} b^3=(a\color{blue} \mathbf{ + } \color{black}b)(a^2\color{red}\mathbf{ - }\color{black}ab\color{darkgreen} \mathbf{+} \color{black}b^2)$\\
		\hline
		&\\
		Difference of two cubes& $a^3\color{blue}\mathbf{ - }\color{black}b^3=(a\color{blue}\mathbf{ - }\color{black}b)(a^2\color{red}\mathbf{ + }\color{black}ab\color{darkgreen} \mathbf{+} \color{black}b^2)$\\
		\hline
	\end{tabular}
\end{center}

\Large
\begin{center}
\begin{tabular}{ccccccc}
\textbf{\color{blue}S\color{black}} &  $\cdot$  &   \textbf{\color{red}O\color{black}}  &   $\cdot$  &   \textbf{\color{darkgreen}A\color{black}}  &   $\cdot$  &  \textbf{ \color{darkgreen}P\color{black}}\\
\begin{turn}{-90}\color{blue}same\color{black}\end{turn} && \begin{turn}{-90}\color{red}opposite\color{black}\end{turn} && \begin{turn}{-90}\color{darkgreen}always\color{black}\end{turn} && \begin{turn}{-90}\color{darkgreen}positive\color{black}\end{turn}

\end{tabular}

\[(a \color{blue}\pm\color{black} b)^3 = (a\color{blue}\pm\color{black}b)(a^2 \color{red}\mp \color{black} ab \color{darkgreen}+\color{black} b^2)\]
\end{center}
\normalsize





\begin{example}
Factor each expression using sum or difference of cubes.
\end{example}

\begin{minipage}[t]{0.45\linewidth}
 (a) $5x^4+40x$
\end{minipage}
\begin{minipage}[t]{0.45\linewidth}
 (b) $8y^3-27$
\end{minipage}
\vfill

\begin{youtry}
Factor each expression using sum or difference of cubes.
\end{youtry}

\begin{minipage}[t]{0.45\linewidth}
 (a) $8+z^6$
\end{minipage}
\begin{minipage}[t]{0.45\linewidth}
 (b) $2x^5-16x^2$
\end{minipage}
\vfill



\vfill
 \noindent\fbox{\large\textbf{6.4 (day 2) Homework}: page 433 \, 10-15 all, 21-31 odds   \small }
%%%%%%%%%%%%%%%%%%%%%%%%%%%%%%%%%%%%%
%%%%%%%%%%%%%%%%%%%%%%%%%%%%%%%%%%%%%
 \newpage
%%%%%%%%%%%%%%%%%%%%%%%%%%%%%%%%%%%%%
%%%%%%%%%%%%%%%%%%%%%%%%%%%%%%%%%%%%%
\rhead{Algebra 2: 6.4 Factoring Review  }
\noindent\Large{\textbf{6.4 Review of Factoring}} \\

\indent\hfill\small\noindent \textbf{Objective}: Factor using the sum and difference of two cubes, difference of square, grouping, and GCF.  \large\\

\noindent Factor using the greatest common factor (GCF).\\

\begin{enumerate}
	\begin{minipage}[t]{0.45\linewidth}
	\item $2x^5-6x^3$ \\
	
	\vspace{2cm}
	\item $5x^3-10x$\\
	
	\vspace{2cm}
	\end{minipage}
	\hfill
	\begin{minipage}[t]{0.45\linewidth}
	\item $14x^3-49x^2-28x$\\
	
	\vspace{2cm}
	\item $27x^5-18x^4+9x^3$\\
	
	\vspace{2cm}
	\end{minipage}
\end{enumerate}

\noindent Factor using difference of squares.\\

\begin{enumerate}
\setcounter{enumi}{4}
	\begin{minipage}[t]{0.45\linewidth}
	\item $q^2-r^2$ \\
	
	\vspace{3cm}
	\item $25a^2-64b^2$\\
	
	\vspace{3cm}
	\item $81x^2-100y^2$\\
	
	\vspace{3cm}	
	\end{minipage}
	\hfill
	\begin{minipage}[t]{0.45\linewidth}
	\item $x^4-y^4$\\
	
	\vspace{3cm}
	\item $a^6-b^6$\\
	
	\vspace{3cm}
	\item $4x^4-9y^6$\\
	
	\vspace{3cm}
	\end{minipage}
\end{enumerate}

%%%%%%%%%%%%%%%%%%%%%%%%%%%%%%%%%%%%%
%%%%%%%%%%%%%%%%%%%%%%%%%%%%%%%%%%%%%
 \newpage
%%%%%%%%%%%%%%%%%%%%%%%%%%%%%%%%%%%%%
%%%%%%%%%%%%%%%%%%%%%%%%%%%%%%%%%%%%%

\noindent Factor using sum and  difference of cubes.\\

\begin{enumerate}
\setcounter{enumi}{4}
	\begin{minipage}[t]{0.45\linewidth}
	\item $x^3-y^3$ \\
	
	\vspace{3cm}
	\item $r^3+s^3$\\
	
	\vspace{3cm}
	\item $8a^3-27b^3$\\
	
	\vspace{3cm}	
	\end{minipage}
	\hfill
	\begin{minipage}[t]{0.45\linewidth}
	\item $64x^3+125y^3$\\
	
	\vspace{3cm}
	\item $a^6-b^6$\\
	
	\vspace{3cm}
	\item $x^6+y^6$\\
	
	\vspace{3cm}
	\end{minipage}
\end{enumerate}

\noindent Factor using grouping.\\

\begin{enumerate}
\setcounter{enumi}{4}
	\begin{minipage}[t]{0.45\linewidth}
	\item $6x^3+2x^2+9x+3$ \\
	
	\vspace{3cm}
	\item $7x^3-35x^2+8x-40$\\
	
	\vspace{3cm}	
	\end{minipage}
	\hfill
	\begin{minipage}[t]{0.45\linewidth}
	\item $4x^3+8x^2-9x-18$\\
	
	\vspace{3cm}
	\item $16x^3-64x^2-25x+100$\\
	
	\vspace{3cm}
	\end{minipage}
\end{enumerate}

\vfill
 \noindent\fbox{\large\textbf{ 6.4 Factoring Review Homework}: page 433 \, 20-30 evens, 33-38 all    \small }

%%%%%%%%%%%%%%%%%%%%%%%%%%%%%%%%%%%%%
%%%%%%%%%%%%%%%%%%%%%%%%%%%%%%%%%%%%%
 \newpage
%%%%%%%%%%%%%%%%%%%%%%%%%%%%%%%%%%%%%
%%%%%%%%%%%%%%%%%%%%%%%%%%%%%%%%%%%%%	

%%%%%%%%%%%%%%%%%%%%%%%%%%%%%%%
%%%%%%%%%%%%%%%%%%%%%%%%%%%%%%%
%%%%%%   Section 6.5   %%%%%%%%
%%%%%%%%%%%%%%%%%%%%%%%%%%%%%%%
%%%%%%%%%%%%%%%%%%%%%%%%%%%%%%%
\rhead{Algebra 2: Section  \getcurrentref{chapter}.\getcurrentref{section}}
 \section{   }
 \indent\hfill\small\noindent \textbf{Objective}: \normalsize\\
 \setcounter{example}{0}
 \setcounter{definition}{0}
 %%%%%%%%%%%%%%%%%%%%%%%%%%%%%%%%
 %%%%%%%%%%%%%%%%%%%%%%%%%%%%%%%%
 \begin{definition}
     Definition
 \end{definition}
 \begin{example}
     Example
 \end{example}

\vfill
 \noindent\fbox{\large\textbf{6.5 Homework}: page \small }
%%%%%%%%%%%%%%%%%%%%%%%%%%%%%%%%%%%%%
%%%%%%%%%%%%%%%%%%%%%%%%%%%%%%%%%%%%%
 \newpage
%%%%%%%%%%%%%%%%%%%%%%%%%%%%%%%%%%%%%
%%%%%%%%%%%%%%%%%%%%%%%%%%%%%%%%%%%%%

%%%%%%%%%%%%%%%%%%%%%%%%%%%%%%%
%%%%%%%%%%%%%%%%%%%%%%%%%%%%%%%
%%%%%%   Section 6.6   %%%%%%%%
%%%%%%%%%%%%%%%%%%%%%%%%%%%%%%%
%%%%%%%%%%%%%%%%%%%%%%%%%%%%%%%
 \section{   }
 \indent\hfill\small\noindent \textbf{Objective}: \normalsize\\
 \setcounter{example}{0}
 \setcounter{definition}{0}
 %%%%%%%%%%%%%%%%%%%%%%%%%%%%%%%%
 %%%%%%%%%%%%%%%%%%%%%%%%%%%%%%%%
 \begin{definition}
     Definition
 \end{definition}
 \begin{example}
     Example
 \end{example}

\vfill
 \noindent\fbox{\large\textbf{6.6 Homework}: page \small }
%%%%%%%%%%%%%%%%%%%%%%%%%%%%%%%%%%%%%
%%%%%%%%%%%%%%%%%%%%%%%%%%%%%%%%%%%%%
 \newpage
%%%%%%%%%%%%%%%%%%%%%%%%%%%%%%%%%%%%%
%%%%%%%%%%%%%%%%%%%%%%%%%%%%%%%%%%%%%

%%%%%%%%%%%%%%%%%%%%%%%%%%%%%%%
%%%%%%%%%%%%%%%%%%%%%%%%%%%%%%%
%%%%%%   Section 6.7   %%%%%%%%
%%%%%%%%%%%%%%%%%%%%%%%%%%%%%%%
%%%%%%%%%%%%%%%%%%%%%%%%%%%%%%%
 \section{   }
 \indent\hfill\small\noindent \textbf{Objective}:  \normalsize\\
 \setcounter{example}{0}
 \setcounter{definition}{0}
 %%%%%%%%%%%%%%%%%%%%%%%%%%%%%%%%
 %%%%%%%%%%%%%%%%%%%%%%%%%%%%%%%%
 \begin{definition}
     Definition
 \end{definition}
 \begin{example}
     Example
 \end{example}

\vfill
 \noindent\fbox{\large\textbf{6.7 Homework}: page \small }
%%%%%%%%%%%%%%%%%%%%%%%%%%%%%%%%%%%%%
%%%%%%%%%%%%%%%%%%%%%%%%%%%%%%%%%%%%%
 \newpage
%%%%%%%%%%%%%%%%%%%%%%%%%%%%%%%%%%%%%
%%%%%%%%%%%%%%%%%%%%%%%%%%%%%%%%%%%%%

%%%%%%%%%%%%%%%%%%%%%%%%%%%%%%%
%%%%%%%%%%%%%%%%%%%%%%%%%%%%%%%
%%%%%%   Section 6.8   %%%%%%%%
%%%%%%%%%%%%%%%%%%%%%%%%%%%%%%%
%%%%%%%%%%%%%%%%%%%%%%%%%%%%%%%
 \section{   }
 \indent\hfill\small\noindent \textbf{Objective}:  \normalsize\\
 \setcounter{example}{0}
 \setcounter{definition}{0}
 %%%%%%%%%%%%%%%%%%%%%%%%%%%%%%%%
 %%%%%%%%%%%%%%%%%%%%%%%%%%%%%%%%
 \begin{definition}
     Definition
 \end{definition}
 \begin{example}
     Example
 \end{example}

\vfill
 \noindent\fbox{\large\textbf{6.8 Homework}: page \small }
%%%%%%%%%%%%%%%%%%%%%%%%%%%%%%%%%%%%%
%%%%%%%%%%%%%%%%%%%%%%%%%%%%%%%%%%%%%
 \newpage
%%%%%%%%%%%%%%%%%%%%%%%%%%%%%%%%%%%%%
%%%%%%%%%%%%%%%%%%%%%%%%%%%%%%%%%%%%%

%%%%%%%%%%%%%%%%%%%%%%%%%%%%%%%
%%%%%%%%%%%%%%%%%%%%%%%%%%%%%%%
%%%%%%   Section 6.9   %%%%%%%%
%%%%%%%%%%%%%%%%%%%%%%%%%%%%%%%
%%%%%%%%%%%%%%%%%%%%%%%%%%%%%%%
 \section{   }
 \indent\hfill\small\noindent \textbf{Objective}: \normalsize\\
 \setcounter{example}{0}
 \setcounter{definition}{0}
 %%%%%%%%%%%%%%%%%%%%%%%%%%%%%%%%
 %%%%%%%%%%%%%%%%%%%%%%%%%%%%%%%%
 \begin{definition}
     Definition
 \end{definition}
 \begin{example}
     Example
 \end{example}

\vfill
 \noindent\fbox{\large\textbf{6.9 Homework}: page \small }
%%%%%%%%%%%%%%%%%%%%%%%%%%%%%%%%%%%%%
%%%%%%%%%%%%%%%%%%%%%%%%%%%%%%%%%%%%%
 \newpage
%%%%%%%%%%%%%%%%%%%%%%%%%%%%%%%%%%%%%
%%%%%%%%%%%%%%%%%%%%%%%%%%%%%%%%%%%%%

\end{document}










